\documentclass{article}

\usepackage{listings}
\usepackage{minted}
\usepackage{hyperref}
\usepackage[a4paper, total={6in, 9in}]{geometry}
\hypersetup{
    colorlinks,
    citecolor=black,
    filecolor=black,
    linkcolor=blue,
    urlcolor=black
}
\begin{document}
\tableofcontents
\newpage
\section{Conquer - Programming documentation}
\subsection{Embedding}
\textit{Conquer} was designed without any GUI in mind. It can run headless and can for example run on a server, or with a GUI made with Swing or JavaFX.
\begin{minted}{java}
XMLReader xmlReader = XMLReader.getInstance();
GlobalContext gc = xmlReader.readInfo();
List<InstalledScenario> scenarios = gc.getInstalledMaps();
Initializer.INSTANCE().initialize(null);
for (InstalledScenario scenario : scenarios) {
	ConquerInfo conquerInstance = ctx.loadInfo(scenario);
	conquerInstance.addContext(gc);
	conquerInstance.init();
	while (!conquerInstance.onlyOneClanAlive()) {
		conquerInstance.executeActions();
	}
}
\end{minted}
This sample code initializes Conquer and plays every installed scenario, until only one clan is alive.\newline
What does each line do?\newline
\begin{minted}{java}
XMLReader xmlReader = XMLReader.getInstance();
\end{minted}
Obtains the singleton-instance for the class \texttt{XMLReader}. It has just one method \texttt{readInfo} that returns the \texttt{GlobalContext} (This means: Installed scenarios,
installed plugins and strategies).
\begin{minted}{java}
Initializer.INSTANCE().initialize(null);
\end{minted}
This line initializes the engine. It may for example update the system properties with properties from a file. \texttt{null} can be a \texttt{Consumer<Exception>} that will be called if an exception 
occurs.
\begin{minted}{java}
ConquerInfo conquerInstance = ctx.loadInfo(scenario);
\end{minted}
Construct the game from the specified scenario.
\begin{minted}{java}
conquerInstance.addContext(context);
\end{minted}
\texttt{addContext} binds the \texttt{GlobalContext} read by \texttt{XMLReader} to \texttt{conquerInstance}.
\begin{minted}{java}
conquerInstance.init();
\end{minted}
\texttt{init} does the final initialization like initializing plugins and binding strategies to clans.
\begin{minted}{java}
conquerInstance.executeActions()
\end{minted}
\texttt{executeActions} is an umbrella for executing plugins, letting the computer play, produce resources and all other things.\newline

The lists returned by \texttt{xmlReader.readInfo()} are mutable, so you are able to add own plugins or strategies as an embedder.\newline
If you want to write an own implementation for the engine of Conquer, you just have to implement \texttt{ConquerInfo}.

\subsection{Plugin API}
A plugin has a lifecycle consisting of four parts:
\begin{itemize}
	\item \textbf{Instantiation} Every plugin should have a default constructor (No arguments). It is instantiated with reflection. If you want to do something only one time, write it into a static initializer. \textbf{Will only be called once while the JVM is running!}
	\item \textbf{Initialization} In this part, \texttt{init(PluginInterface)} will be called. In this method it is allowed to register callback functions. 
	\item \textbf{Round} Every round, the method \texttt{handle(Graph<City>,Context)} will be called. It is undefined, when it is invoked (Before anything else or at the end of the round).
	\item \textbf{Exit} In this part, \texttt{exit(Result)} will be called. Use this method for cleanup (Closing file descriptors, streams, write data,...).
	
\end{itemize}
\subsubsection{Callbacks}
\paragraph{\texttt{AttackHook}s} are called when an attack is executed. The interface has two methods: \texttt{before(City,City,long)} and \texttt{after(City,City,long,AttackResult}). The first one will
be called as soon as the number of soldiers are determined. The latter one after the attack was executed, clan changes were made and soldiers killed. It contains the result as \texttt{AttackResult}.
\paragraph{\texttt{MessageListener}s} are called as soon as a message was added to or removed from the EventList. It allows you to react to a huge number of different events like changed relationships, random events and anything else.
\paragraph{\texttt{MoneyHook}s} will be called after every city generated the coins and all cities did generated them.
\paragraph{\texttt{MoveHook}s} are executed when soldiers are moved. (Between cities of the same clan).  
\paragraph{\texttt{RecruitHook}s} allow you to get notified when soldiers are recruited.
\paragraph{\texttt{ResourceHook}s} are executed for each city after resources were produced.\newline

If you combine this callbacks, you are able to access the state of the game nearly every time it changes.
\subsection{Strategy API}
\subsubsection{\texttt{StrategyProvider}}
Only an instance of \texttt{StrategyProvider} is instantiated initially. It must have a default constructor, otherwise it instantiation may fail. \texttt{StrategyProvider} has two important methods:
\begin{minted}{java}
Strategy buildStrategy();
\end{minted}
This method returns a new \texttt{Strategy} object that will be used by a clan. A different object has to be returned every time this method is invoked, otherwise some weird things may happen.
\begin{minted}{java}
byte getId();
\end{minted}
This returns an unique, non-negative identifier for this strategy. If two StrategyProviders return the same identifier, the used strategy is undefined.
\subsubsection{\texttt{Strategy}}
\texttt{Strategy} has three important methods:
\begin{minted}{java}
void applyStrategy(Clan clan, Graph<City> cities, StrategyObject obj);
\end{minted}
This method is called every round. \texttt{clan} is the clan for which the strategy should play. \texttt{cities} are all cities in the scenario and  \texttt{obj} is responsible for executing
actions like attacks or upgrades.
\begin{minted}{java}
boolean acceptGift(IClan sourceClan, IClan destinationClan,
			Gift gift, double oldValue,
			DoubleConsumer newValue,
			StrategyObject strategyObject);
\end{minted}
This method is called when one clan wants to give a gift to the clan of the strategy. If the strategy accepts the gift, \texttt{newValue} has to be called with the new relationship value. For example, if the relationship
was at 50 points and after that it should be 57.5, run \texttt{newValue.accept(57.5);} instead of \texttt{newValue.accept(7.5);}. Furthermore \texttt{true} has to be returned if the gift was accepted.
If the strategy rejected the gift, \texttt{false} has to be returned and \texttt{newValue::accept} mustn't be called.

\begin{minted}{java}
StrategyData getData();
\end{minted}
This returns an optional wrapper for everything the strategy needs as storage. This method may return null.
\subsubsection{\texttt{StrategyData}}
This is an optional piece of data that may be used for holding all required values. This interface only provides one method:
\begin{minted}{java}
void update(int currentRound);
\end{minted}
This method is called after the clan for the parent \texttt{Strategy} played.

\subsection{Messages}
Messages can be seen like an event. You can add a \texttt{MessageListener} and get notified as soon as any message was fired. This can be used for example to listen for attacks, troop movements or random events.
\texttt{Message} has several important methods:
\begin{minted}{java}
String getMessageText();
\end{minted}
This should return a (preferably localized) string with the formatted text. This text may be used for showing it in e.g. an event log.
\begin{minted}{java}
boolean isBadForPlayer();
\end{minted}
Only used for showing, e.g. a bad message for the player is in red, while a good one is in green.
\begin{minted}{java}
boolean isPlayerInvolved();
\end{minted}
Allows to sort out the unrelated und for the player useless news.
\begin{minted}{java}
boolean shouldBeShownToThePlayer();
\end{minted}
Allows to sort out messages that are either debug messages or information that would give the player an unfair advantage.

\newpage

\section{Conquer - Interfaces}

In order to make compatibility between different engines possible, a 
set of interfaces with specified behavior is required.

\subsection{ConquerInfo}

\subsubsection{\texttt{void addContext(GlobalContext)}}
A conforming implementation shall throw an \texttt{IllegalArgumentException}, if the context is \texttt{null}.
Furthermore an \texttt{IllegalArgumentException} shall be thrown if the list
of plugins, strategies or installed maps is \texttt{null}.

\subsubsection{\texttt{int currentRound()}}
This method shall only return a non-negative integer.

\subsubsection{\texttt{void exit(Result)}}
A conforming implementation must throw an \texttt{IllegalArgumentException},
if the result is \texttt{null}.

\subsubsection{\texttt{Image getBackground()}}
This method may return \texttt{null}.

\subsubsection{\texttt{IClan getClan(int)}}
An \texttt{IllegalArgumentException} shall be thrown, if the given
id is out of bounds.

\subsubsection{\texttt{List<String> getClanNames()}}
This method should return an unmodifiable list of clan names, sorted by
ascending id.

\subsubsection{\texttt{List<IClan> getClans()}}
This method must return the original list of clans, so modifications are mirrored at the internal state.

\subsubsection{\texttt{List<Double> getCoins()}}
This method should return an unmodifiable list of coins every clan has, sorted by
ascending id.

\subsubsection{\texttt{List<Color> getColors()}}
This method should return an unmodifiable list of color an clan is associated with, sorted by
ascending id.

\subsubsection{\texttt{List<Plugin> getPlugins()}}
This method shall return the original non-null list of plugins in no specified order.

\subsubsection{\texttt{ConquerSaver getSaver(String)}}

An \texttt{IllegalArgumentException} shall be thrown, if the given
string is either \texttt{null} or \texttt{string.isEmpty()} is true.

\subsubsection{\texttt{boolean isDead(IClan)}}
An \texttt{IllegalArgumentException} shall be thrown, if \texttt{null} is passed.

\subsubsection{\texttt{long maximumNumberOfSoldiersToRecruit(IClan,long)}}
Returns the maximum number of soldiers that could be recruited in a clan. This value may never be \texttt{null}.
An \texttt{IllegalArgumentException} shall be thrown if the given clan is either \texttt{null} or the given long is negative.

\subsubsection{\texttt{void setPlayerGiftCallback(PlayerGiftCallback)}}
The argument shall never be \texttt{null}.
If it is \texttt{null}, an \texttt{IllegalArgumentException} has to be thrown.

\subsubsection{\texttt{void upgradeDefenseFully(ICity)}}
The given city mustn't be \texttt{null}.
Otherwise an \texttt{IllegalArgumentException} must be thrown.

\subsubsection{\texttt{void upgradeResourceFully(Resource,ICity)}}
The given city and the resource mustn't be \texttt{null}.
Otherwise an \texttt{IllegalArgumentException} must be thrown.

\subsubsection{\texttt{void upgradeSoldiersDefenseFully(IClan)}}
The given clan may not be \texttt{null}, otherwise an \texttt{IllegalArgumentException}
shall be thrown.

\subsubsection{\texttt{void upgradeSoldiersFully(IClan)}}
The given clan may not be \texttt{null}, otherwise an \texttt{IllegalArgumentException}
shall be thrown.

\subsubsection{\texttt{void upgradeSoldiersOffenseFully(IClan)}}
The given clan may not be \texttt{null}, otherwise an \texttt{IllegalArgumentException}
shall be thrown.

\subsection{PluginInterface}

\subsubsection{\texttt{void addAttackHook(AttackHook)}}
The argument may never be \texttt{null}. If it is \texttt{null}, an \texttt{IllegalArgumentException}
shall be thrown.

\subsubsection{\texttt{void addCityKeyHandler(String,CityKeyHandler)}}
The arguments may never be \texttt{null}. If one of them is \texttt{null}, an \texttt{IllegalArgumentException}
shall be thrown.

\subsubsection{\texttt{void addKeyHandler(String,KeyHandler)}}
The arguments may never be \texttt{null}. If one of them is \texttt{null}, an \texttt{IllegalArgumentException}
shall be thrown.

\subsubsection{\texttt{void addMessageListener(MessageListener)}}
The argument may never be \texttt{null}. If it is \texttt{null}, an \texttt{IllegalArgumentException}
shall be thrown.

\subsubsection{\texttt{void addMoneyHook(MoneyHook)}}
The argument may never be \texttt{null}. If it is \texttt{null}, an \texttt{IllegalArgumentException}
shall be thrown.

\subsubsection{\texttt{void addMoveHook(MoveHook)}}
The argument may never be \texttt{null}. If it is \texttt{null}, an \texttt{IllegalArgumentException}
shall be thrown.

\subsubsection{\texttt{void addMusic(String)}}
The argument may never be \texttt{null}. If it is \texttt{null}, an \texttt{IllegalArgumentException}
shall be thrown.

\subsubsection{\texttt{void addRecruitHook(RecruitHook)}}
The argument may never be \texttt{null}. If it is \texttt{null}, an \texttt{IllegalArgumentException}
shall be thrown.

\subsubsection{\texttt{void addResourceHook(ResourceHook)}}
The argument may never be \texttt{null}. If it is \texttt{null}, an \texttt{IllegalArgumentException}
shall be thrown.

\subsubsection{\texttt{EventList getEventList()}}
The valid, non-null \texttt{EventList} must be returned.

\subsection{ConquerInfoReader}

\subsubsection{\texttt{ConquerInfo build()}}

Any exception may be thrown. 

\subsection{ConquerInfoReaderFactory}

\subsubsection{\texttt{ConquerInfoReader getForFile(String)}}

The given String may not be \texttt{null}, otherwise an \texttt{IllegalArgumentException} shall be thrown.
The returned object may never be \texttt{null}.

\subsubsection{\texttt{byte[] getMagicNumber()}}
The returned byte-array must be non-null.

\subsection{ConquerSaver}

Any class implementing this interface must have an accessible constructor taking a String parameter
with the name of the saved name.

\subsubsection{\texttt{ConquerInfo restore() throws Exception}}
Any exception may be thrown. \texttt{null} may never be returned.

\subsubsection{\texttt{void save(ConquerInfo) throws Exception}}
The given object may not be \texttt{null}. Otherwise an \texttt{IllegalArgumentException} may be thrown.

\subsection{PlayerGiftCallback}

\subsubsection{\texttt{boolean acceptGift(IClan,IClan,Gift,double,...)}}
\begin{minted}{java}
boolean acceptGift(IClan source, IClan destination, Gift gift,
			double oldValue,
			DoubleConsumer newValue,
			StrategyObject strategyObject);
\end{minted}
You can make several assumptions:
\begin{itemize}
	\item None of this values may be \texttt{null}.
	\item The clan \texttt{destination} is always the clan of the player.
	\item \texttt{oldValue} is in range $]0;100[$
\end{itemize}

If the player accepted the gift, \texttt{newValue} has to consume the new relationship value and
\texttt{true} has to be returned.
If \texttt{false} is returned, \texttt{newValue} mustn't be called.

\subsection{ICity}

\subsubsection{\texttt{double getBonus()}}
The returned value must always be greater equals 1.

\subsubsection{\texttt{IClan getClan()}}
The returned value mustn't be \texttt{null}

\subsubsection{\texttt{int getClanId()}}
The clan id must be non-negative.

\subsubsection{\texttt{double getDefense()}}
The base defense that is returned must be non-negative.

\subsubsection{\texttt{double getDefenseStrength()}}
The defense strength of a city is always positive or zero.

\subsubsection{\texttt{double getDefenseStrength(IClan)}}
The argument mustn't be \texttt{null}, otherwise an \texttt{IllegalArgumentException} shall be thrown.
The returned value is always positive or zero.

\subsubsection{\texttt{double getGrowth()}}
The returned value must be positive, and should be in the range $]0.75;1.25[$.

\subsubsection{\texttt{Image getImage()}}
The returned value may be \texttt{null}.

\subsubsection{\texttt{ConquerInfo getInfo()}}
The returnvalue must be non-null.

\subsubsection{\texttt{List<Integer> getLevels()}}
The list that was returned mustn't be \texttt{null}. Every element in it
should be non-null, too and have a positive value.
The length of the list should be equals to the number of resources plus one.
(\texttt{Resource.values().length + 1}, the additional value is for the defense)

\subsubsection{\texttt{long getNumberOfPeople()}}
The return value must be positive or zero.

\subsubsection{\texttt{long getNumberOfSoldiers()}}
The return value must be positive or zero.

\subsubsection{\texttt{List<Double> getProductions()}}
The list that was returned must be non-null. Every element in it has to be non-null,
too and have a positive value.
The length of the list should be equals to the number of resources.

\subsubsection{\texttt{int getX()}}
The return value must be positive or zero.

\subsubsection{\texttt{int getY()}}
The return value must be positive or zero.

\subsubsection{\texttt{double productionPerRound(Resource)}}
The argument may never be \texttt{null}, otherwise an \texttt{IllegalArgumentException} is thrown.
The returned value must be positive or zero.

\subsubsection{\texttt{void setClan(IClan)}}
The argument may never be \texttt{null}, otherwise an \texttt{IllegalArgumentException} is thrown.

\subsubsection{\texttt{void setDefense(double)}}
The argument may never be negative, otherwise an \texttt{IllegalArgumentException} is thrown.

\subsubsection{\texttt{void setGrowth(double)}}
The argument may never be negative, otherwise an \texttt{IllegalArgumentException} is thrown.

\subsubsection{\texttt{void setNumberOfPeople(long)}}
The argument may never be negative, otherwise an \texttt{IllegalArgumentException} is thrown.

\subsubsection{\texttt{void setNumberOfSoldiers(long)}}
The argument may never be negative, otherwise an \texttt{IllegalArgumentException} is thrown.

\subsection{IClan}

\subsubsection{\texttt{double getCoins()}}
The returned value must be positive or zero.

\subsubsection{\texttt{Color getColor()}}
The returned value mustn't be \texttt{null}

\subsubsection{\texttt{StrategyData getData()}}
StrategyData is optional. This may result in the return value being \texttt{null}.

\subsubsection{\texttt{int getId()}}
The id of a clan is an unique, positive or zero identifier.

\subsubsection{\texttt{String getName()}}
The name may never be \texttt{null}.

\subsubsection{\texttt{List<Double> getResources()}}
The list mustn't be \texttt{null}. Every element must be positive or zero.
The length of the list is equals to the number of resources. (\texttt{Resource.values().length})

\subsubsection{\texttt{List<Double> getResourceState()}}
The list mustn't be \texttt{null}. Every element must be positive or zero.
The length of the list is equals to the number of resources. (\texttt{Resource.values().length})

\subsubsection{\texttt{int getSoldiersXXXLevel()}}
The returned value has to be greater equals zero.

\subsubsection{\texttt{double getSoldiersXXXStrength()}}
The returned value must be greater equals one.

\subsubsection{\texttt{Strategy getStrategy()}}
The returned strategy may never be \texttt{null}.

\subsubsection{\texttt{void init(StrategyProvider[],Version)}}
None of the arguments may be \texttt{null}. Otherwise an \texttt{IllegalArgumentException} will be thrown.

\subsubsection{\texttt{void setColor(Color)}}
An \texttt{IllegalArgumentException} will be thrown, if the argument is \texttt{null}.
After setting the color, calling this method again will result in an \texttt{UnsupportedOperationException}.

\subsubsection{\texttt{void setId(int)}}
An \texttt{IllegalArgumentException} will be thrown, if the argument is negative.
After setting the id, calling this method again will result in an \texttt{UnsupportedOperationException}.

\subsubsection{\texttt{void setName(String)}}
An \texttt{IllegalArgumentException} will be thrown, if the argument is \texttt{null}.
After setting the name, calling this method again will result in an \texttt{UnsupportedOperationException}.

\subsubsection{\texttt{void setResources(List<Double>)}}
An \texttt{IllegalArgumentException} will be thrown, if the argument is either \texttt{null}
or the length of the list is not equals to the number of resources.
A second call to this method with valid arguments will result in an \texttt{UnsupportedOperationException}.

\subsubsection{\texttt{void setResourceStats(List<Double>)}}
An \texttt{IllegalArgumentException} will be thrown, if the argument is either \texttt{null}
or the length of the list is not equals to the number of resources.
A second call to this method with valid arguments will result in an \texttt{UnsupportedOperationException}.

\subsubsection{\texttt{void setStrategy(Strategy)}}
An \texttt{IllegalArgumentException} will be thrown, if the argument is negative.
After setting the strategy, calling this method again will result in an \texttt{UnsupportedOperationException}.
This method should only be used to restore a clan after saving it, otherwise \texttt{init} should be used.

\subsubsection{\texttt{void update(int)}}
The argument must be positive or zero. Otherwise an \texttt{IllegalArgumentException} will be thrown.

\subsection{Strategy}

\subsubsection{\texttt{boolean acceptGift(IClan,IClan,Gift,double,...)}}
\begin{minted}{java}
boolean acceptGift(IClan sourceClan, IClan destinationClan, Gift gift,
                        double oldValue,
                        DoubleConsumer newValue,
                        StrategyObject strategyObject);
\end{minted}
\texttt{destinationClan} is always the clan of this strategy. No argument may ever be \texttt{null}.

\subsubsection{\texttt{void applyStrategy(IClan,Graph<ICity>,StrategyObject)}}
No argument can be \texttt{null}. The given clan is the clan for this strategy.

\subsubsection{\texttt{StrategyData getData()}}
\texttt{null} may be returned.

\subsubsection{\texttt{StrategyData resume(StrategyObject,byte[],boolean,byte[])}}
\begin{minted}{java}
StrategyData resume(final StrategyObject strategyObject,
			final byte[] bytes,
			final boolean hasStrategyData,
			final byte[] dataBytes)
\end{minted}
\texttt{strategyObject} and \texttt{bytes} may never be \texttt{null}.
The value of \texttt{dataBytes} is undefined, if \texttt{hasStrategyData} equals \texttt{false}.
Otherwise this array is non-null.
If \texttt{hasStrategyData} is \texttt{true}, a non-null object has to be returned, \texttt{null} otherwise.

\subsubsection{\texttt{void save(OutputStream) throws IOException}}
Only save the state of the strategy. The state of the StrategyData, if available, will be saved by itself.
The stream is never \texttt{null} and may not be closed.


\end{document}
